% paper.tex
% sample ACM SIG Proceedings document using LaTeX2e
% Author: Heila van der Merwe
% based upon LaTeX2.09 Guidelines, 9 June 1996
% Revisions:  10 July 2012

%


\documentclass{acm_proc_article-sp}
\usepackage{graphicx}
\begin{document}


\title{Verifying Android Applications using Java PathFinder}

\numberofauthors{3}

\author{%
\alignauthor  Brink van der Merwe\\
 \affaddr{Dept. of Computer Science}\\
       \affaddr{University of Stellenbosch}\\
       \affaddr{Private Bag X1 Matieland}\\
       \affaddr{South Africa, 7602}
       \email{abvdm@gmail.com}
\alignauthor Heila van der Merwe\\
       \affaddr{Dept. of Computer Science}\\
       \affaddr{University of Stellenbosch}\\
       \affaddr{Private Bag X1 Matieland}\\
       \affaddr{South Africa, 7602}
       \email{heilamvdm@gmail.com}
\alignauthor  Willem Visser\\
 \affaddr{Dept. of Computer Science}\\
       \affaddr{University of Stellenbosch}\\
       \affaddr{Private Bag X1 Matieland}\\
       \affaddr{South Africa, 7602}
       \email{willem@gmail.com}
}


\maketitle

\begin{abstract}


\end{abstract}

% A category with the (minimum) three required fields
\category{H.4}{Information Systems Applications}{Miscellaneous}
%A category including the fourth, optional field follows...
\category{D.2.8}{Software Engineering}{Metrics}[complexity measures, performance measures]

\terms{VERIFICATION}

\keywords{Java PathFinder, JPF, Android, Verification} % NOT required for Proceedings

\section{Introduction}
%The current trend in software development is mobile application development.

%At Google I/O 2012 Android claimed to have approximately 600
%thousand applications on the market with a total of 20 billion application downloads. They continue to note that this number is growing by
%one billion per month.

Although Android mobile applications and Java desktop applications both use the Java language, they are developed for a completely different
platforms.  and both contain many similar errors and defects. These defects include faults such as concurrency issues and run-time
exceptions. 
         are  compared to
         
         
The testing of Android applications  also faces many challenges including:
\begin{enumerate}
 \item The device challenge. 
 Mobile applications can only fully run on a mobile devices/emulators. Additionally the physical devices are
not designed with testing in mind.

 \item Event driven architecture.
 The event input must be simulated which is especially difficult due to the many types of
sensory inputs that must be supported. 

 \item Operating system variety. It is impossible to test the application on all the different operating system versions.
\end{enumerate}

Due to these challenges and the rapid pace at which applications are developed and released, application testing becomes too expensive in
terms of time and money. This results in many mobile applications being unstable and unreliable.

Manual testing is not the solution to the problem as it is very expensive, time consuming and does not detect enough errors in
Android applications. That is why only automatic testing frameworks are the best options.

There are currently two ways to automatically test Android applications:
\begin{enumerate}
 \item testing on a physical device/emulator
  Install application on the device. Install test project that runs in the same process. The test project drives the application testing. 
 

 
 %\begin{figure}[h!]
 %\centering
 %\includegraphics{../../Pictures/test_framework.png}
 % test_framework.png: 473x517 pixel, 96dpi, 12.51x13.68 cm, bb=0 0 355 388
%\end{figure}
  Examples include: Robotium\ref{} Android Test project

  MoneyRunner~\ref{},SPark~\ref{}
  -advantage wifi and many sensor events  already modelled
-problem is that emulator run actual android so testing framework as to either be android application
-
 
 \item testing the application on a computer
   Examples:
    android unit tests
    android lint
    
   mock objects
   adapt for pc
- static analysis or model android framework
- mock unit tests - still have to write the tests
- Robolectric
\end{enumerate}
 

% what are we going to do and why (WHAT ARE ADVANTAGES)
This paper describes jpf-android, an extension to Java PathFinder that allows the automatic testing of Android Applications. 

The first section of the paper provides an overview of Android and Java Pathfinder and how JPF can be used to test Android applications.
The next section describes the design and development of the jpf-android tool followed by an simple example to show how the tool works. The
paper will conclude by discussing the effectiveness and usefulness of the tool.








\newpage
\section{Background}
\subsection{The Android architecture}
Android is an open source software stack for mobile devices such as smartphones and tablets with an advanced RISC Machine (ARM)
architecture.

It consists of:
    the Android operating system built on top of a modified Linux kernel,
    the application framework - set of libraries api
    an application development toolkit assisting developers to create applications for the platform.

    
An application runs on framework using api libs to communicate.
system process threading

%\begin{figure}[h!]
% \centering
% \includegraphics[width=\pagewidth]{../../M-Projek/heila/research_proposal/02-AndroidStack.png}
 % 5502982451_1f78830c9e.jpg: 500x270 pixel, 96dpi, 13.23x7.14 cm, bb=0 0 375 203
%\end{figure}

each application runs in own process
1 thread called the looper
handler message queue structure
2 types of input


%\begin{figure}[h!]
% \centering
% \includegraphics[width=\pagewidth]{../../M-Projek/heila/research_proposal/5502982451_1f78830c9e.jpg}
 % 5502982451_1f78830c9e.jpg: 500x270 pixel, 96dpi, 13.23x7.14 cm, bb=0 0 375 203
%\end{figure}

% why cant we just run android applications they are written in java?
Specific java version without swing etc
native code - has to be modelled


\subsection{Java PathFinder}
JPF is a powerful software model checker used to test automatically whether the system under test (SUT) meets a given specification.
Java applications~\cite{}. It consists of a custom Java virtual machine implementation that executes the compiled java byte-code of the
system under test(SUT).


JPF models the Java API libraries to allow... 
-how does it work?
-modelled java libs
-custom vm executing java bytecode
-controlled environment executing all execution paths
-listeners for certain property violations 

Furthermore, JPF's design encourages developers to create extensions to the framework. It provides that key mechanisms:
to model java libraries,
model native method calls (MJI),
track bytecode execusion and listen for property violations


Currently there exists many extensions including a symbolic execution extension (jpf-symbc), data race detector (jpf-racefinder) and a
abstract window toolkit (AWT) extension (jpf-awt).

One of the advantages of extending JPF is that it has been thoroughly tested and already works for Java applications. So as soon as the the
Android applications can run on JPF, common errors and defects can be detected.

\subsection{Applying JPF to Android applications}
Android applications are written in a custom Java implementation Apache Harmony.

Dalvik, the virtual machine used in Google's Android platform, uses a subset of Harmony for the core of its Class Library.[15] However,
Dalvik does not align to Java SE nor Java ME Class Library profiles (for example J2ME classes, AWT and Swing are not supported). Instead it
uses its own library,[16] built on a subset of Harmony.

-model the android framework so that android applications can run on the java JVM
-on it can be run on the java jvm we can run it on jpf’s java jvm

- testing with object maps
- advantage above capture and replay is that all event sequences are executed


%what do I want to detect?

The objective of this project is to create a JPF extension based on JPF-AWT's scripting mechanism that will:
\begin{itemize}
 \item model the Android framework libraries using Java SE. Android applications are built on a framework written in Apache Harmony's Java
implementation. This framework will basically be simplified and closed down to allow the Android application to execute in a controlled
environment of JPF on a computer.

 \item model user and system input to the application. Android applications are event driven. We need to create a user model to simulate
input events for the application, similar to a driver. 

 \item use these models to automatically detect property violations such as deadlocks,race conditions and run-time exceptions in Android
applications.

\end{itemize}
\newpage
\section{Development}
% what are the issues of modelling the framework?
size of android framework - what to model
  - we don't necessarily want to have to run android applications exactly as they are run, then gets to complication to test all application
  paths.

   - the more is modelled the more has to be re-modelled if something changes

   - if too little is modelled, then support to little functionality

native code and the fact that it is run in different thread - but if not native cannot be back tracked

2 important parts to model:
- processes we don’t need ipc on computer everything runs in one process and we do not need this for testing
- binder ipc communication between components very import part see application as a whole
- gui\& system events drives applciation
- application framework


% description of the model:
1 process
1 main thread
1 application
direct communication instead of binder
no gui drawing

\subsection{The model}
- how is an application structured
--activitythread
--messagequeue + handler structure
--window structure


script input
- what is the purpose of the script
-- input to program
-- vars added
-- later check added

- what can the script handle
- GUI and external events
-- depth first
-- non-determinism

- what can the script not handle
-- specific activity event structure
-- progressive disclosure

how and where is the script events handled
- messagequeue processScriptAction
- generates event sequences uses backtrack to do depth first search trough all possible sequences
\newpage
\section{Example}
% example showing the detection of a null pointer exception in certain case





\section{Discussion}
%discussion showing how the tool is successful and the advantages of the tool




\section{Conclusion}


limitations
- only 1 application
- no android native code can be handled



\section{Acknowledgements}

%
% The following two commands are all you need in the
% initial runs of your .tex file to
% produce the bibliography for the citations in your paper.
\bibliographystyle{abbrv}
\bibliography{paper}  % sigproc.bib is the name of the Bibliography in this case
% 
\end{document}
